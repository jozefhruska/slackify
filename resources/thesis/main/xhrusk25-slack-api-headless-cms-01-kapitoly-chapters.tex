%---------------------------------
% Úvod
%---------------------------------
\chapter{Úvod}
Tradičné redakčné systémy pre správu obsahu sú obvykle zostavené z dvoch hlavných súčastí -- administračného a verejného webového rozhrania. Administračné rozhranie slúži pre vytváranie a úpravu obsahu, webové na jeho nasledné zobrazenie. Webové rozhranie je obvykle jednotné pre všetky platformy a zariadenia na ktorých je využívané, tzn. že jeho používanie je často neoptimálne. Pre tento dôvod sa začali využívať systémy bez webového rozhrania disponujúce klasickým administračným rozhraním a otvoreným aplikačným rozhraním (API\footnote{API - Application programming interface}), umožnujúcim získavanie obsahu na požadované platformy. Tento obsah je následne možné optimalizovať individuálne podľa potreby. Súčasné riešenia redakčných systémom s otvoreným rozhraním sú často robustné systémy, ktoré však vyžadujú komplexnú konfiguráciu predtým ako ich je možné začať využívať. Niektoré takéto systémy vyžadujú aj vlastnú infraštruktúru pre ich nasadenie. Tieto skutočnosti otvárajú priestor pre redakčný systém, ktorý by pomohol vyriešiť tieto prekážky. Takýto systém je cieľom tejto práce.

Navrhovaný systém poskytuje možnosť správy obsahu bez nutnosti úvodnej konfigurácie a infraštruktúry. Obsah je možné spravovať pomocou užívateľského rozhrania v aplikácii Slack alebo webového rozhrania. Plnú funkcionalitu však užívateľ môže využiť bez nutnosti používania webového rozhrania.

\subsection*{Obsah kapitol}
Prvá kapitola sa venuje teoretickej časti a skladá sa z troch sekcií venovaným postupne existujúcim riešeniam podobným tejto práci, vývoji serverových následne klientských aplikácii.

% TODO: Add introduction to document's structure.

%---------------------------------
% Teoretická časť
%---------------------------------
\chapter{Teoretická časť}
Tvorba obsahu a jeho doručenie konzumentom. Táto kapitola popisuje základné princípy, techniky a technológie, ktoré sú nutné pre zostavenie a pochopenie princípu fungovania tejto práce.
\section{Redakčné systémy s otvoreným aplikačným rozhraním}

\section{Vývor serverových aplikácii}
\section{Vývoj klientských aplikácii}
