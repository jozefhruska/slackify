% Úvod
%---------------------------------------------------------------------------
\chapter{Úvod}
Tradičné redakčné systémy pre správu obsahu sú obvykle zostavené z~dvoch hlavných súčastí -- \emph{administračného} a \emph{verejného webového rozhrania}. Administračné rozhranie slúži pre tvorbu a úpravu obsahu, webové na jeho nasledné zobrazenie. Webové rozhranie je typicky jednotné pre všetky typy zariadení na ktorých je zobrazované, tzn. že jeho používanie často nie je optimálne (napr. na mobilných telefónoch s dotykovou obrazovkou).

Pre tento dôvod sa začali využívať systémy bez webového rozhrania, disponujúce klasickým administračným a \emph{otvoreným aplikačným rozhraním} (API\footnote{API -- Application programming interface}), ktoré umožňuje obsah získať a následne ho zobraziť optimálne na ľubovoľnej platforme. Takéto redakčné systémy sa nazývajú \emph{Headless CMS}. Súčasné Headless CMS sú často robustné systémy, ktoré však vyžadujú komplexnú konfiguráciu predtým, ako ich je možné začať využívať. Niektoré takéto systémy potrebujú aj vlastnú infraštruktúru pre ich nasadenie. Tieto skutočnosti otvárajú priestor pre taký Headless CMS, ktorý by pomohol vyriešiť tieto prekážky. Takýto redakčný systém je cieľom tejto práce.

Navrhovaný systém poskytuje možnosť správy obsahu bez nutnosti úvodnej konfigurácie a infraštruktúry. Obsah je možné spravovať pomocou užívateľského rozhrania v~aplikácii Slack alebo doplnkového webového rozhrania.

\subsection*{Obsah kapitol}
% TODO: Add introduction to document's structure.

% Vývoj klientských aplikácii
%---------------------------------------------------------------------------
\chapter{Vývoj klientských aplikácii}
\label{theory:client_dev}
Zobrazenie obsahu užívateľom. Kapitola popisuje teoretické znalosti nutné pre návrh a implementáciu klientských aplikácií (v~prípade tejto práce webovej stránky a Slack aplikácie). Webové technológie sa vyvíjajú vysokou rýchlosťou a s~nimi aj nároky užívateľov na rýchlosť, použiteľnosť, ale aj vzhľad takýchto aplikácií.

Sekcia \ref{theory:UX} sa venuje základným pravidlám user experience\footnote{user experience -- uživateľská skúsenosť}, sekcie \ref{theory:HTML} až \ref{theory:typescript} približujú programovacie a značkovacie jazyky, ktoré sú použité v tejto práci. Posledné sekcie \ref{theory:react} a \ref{theory:nextjs} popisujú dve hlavné knižnice využívané pre vývoj moderných webových aplikácií React a Next.js.

% User Experience (UX)
\section{User experience (UX)}
\label{theory:UX}
Prí návrhu uživateľského rozhrania je jedným z~najdôležiteľších parametrov \emph{uživateľská skúsenosť}. Dizajnéri sa snažia zaistiť aby ich návrh bol intuitívny, jednoduchý, ale zároveň aj plne použiteľný a originálny. Pri UX analýze sa dizajnér snaží vnímať svoj návrh zo strany koncového užívateľa. Vo svete však neexistuje jednotná definícia úkonov, ktoré vedú k~dokonalému uživateľskému zážitku.

\subsection{Porozumenie potrebám užívateľa}
Dizajnér sa pri návrhu pozerá na produkt ako jeho koncový užívateľ. Analyzuje potreby, pre ktoré sa uživateľ rozhodol produkt využívať, ale aj problémy, ktoré užívateľovi bránia v jednoduchom a intuitívnom používaní daného produktu. Po porozumení potrieb užívateľa príchádza z riešeniami, ktoré by však nemali vytvoriť ďaľšie problémy a prekážky v používaní produktu.

\subsection{Prístupnosť (Accessibility)}
Veľmi jednoducho dosiahnuteľná, ale často zanedbávaná vlastnosť je dobrá prístupnosť (použiteľnosť) pre ľudí so zdravotnými znevýhodneniami. Dizajnér musí zaistiť, aby farebné pozadia jednotlivých prvkov mali dostatočný kontrast od ich obsahu,alebo zvýrazniť prvok v~prípade, že je užívateľom (túto vlastnosť majú v~určitej podobe vstavané aj niektoré webové prehliadače, avšak nie vždy optimálne).

% HTML
\section{HTML}
\label{theory:HTML}
HTML (Hypertext Markup Language) je \emph{značkovací jazyk}, pomocou ktorého je možné popísať štruktúru webových stránok. Skladá sa zo stromu elementov [\ref{theory:HTML:elements}], ktoré majú svoj obsah, parametre a sú ohraničené pomocou značiek (tags).

\blockquote[MDN \cite{MDN}]{HTML je nejzákladnejším stavebným kameňom webových stránok. \uv{Hypertext}, v~názve odkazuje k~možnosti vytvorenia odkazov, ktorými je možné prepojiť webové stránky.}

\noindent Pre zobrazovanie HTML slúžia webové prehliadače. Každý webový prehliadač postupuje pri vykresľovaní niektorých častí HTML inak ako ostatné, preto je nutné skontrolovať či je webová stránka správne zobrazená na viacerých webových prehliadačoch.

\subsection{Elementy}
\label{theory:HTML:elements}
HTML element je oddelený od zbytku textu v~dokumente pomocou značiek (tags), ktoré pozostávajú z~názvu elementu ohraničeného znakmi \uv{\texttt{<}} a \uv{\texttt{>}}. Názov elementu vo vnútri značky je \emph{case insensitive}, tzn. že nezáleží či je písaný veľkými alebo malými písmenami. Napríklad značka \texttt{<title>} môže byť napísaná aj ako \texttt{<Title>} alebo \texttt{<TITLE>}. Všetky tieto zápisy sú správne. \cite{MDN} \\

\noindent Značky sa klasifikujú na dve skupiny -- \emph{párové} a \emph{nepárové}. Párové značky sú také, ktoré obsah elementu ohraničujú otvárajúcou (\texttt{<title>}) a uzatvárajúcou (\texttt{</title>}) značkou. Nepárové značky sú také, ktoré nemajú svoju uzatvárajúcu značku, napríklad obrázok (\texttt{<img />}). \\

\noindent Zoznam niektorých najpoužívanejších elementov:
\begin{itemize}
	\item \texttt{head} -- Obsahuje strojom čítateľné informácie (metadáta) o~dokumente ako napríklad titulok, skripty alebo štýly. \cite{MDN}
	\item \texttt{body} -- Reprezentuje obsah HTML dokumentu, pričom sa v~jednom dokumente môže nachádzať maximálne raz. \cite{MDN}
	\item \texttt{title} -- Definuje titulok dokumentu, ktorý je zobrazený vo webovom prehliadači. \cite{MDN}
	\item \texttt{button} -- Reprezentuje klikateľné tlačidlo, použiteľné napríklad pre potvrdzovanie formulárov alebo kdekoľvek inde v~HTML dokumente ako štandardné tlačidlo. Tlačidlá sú v~štýle jednotnom s~platformou na ktorej sú zobrazované, ak nie sú priložené štýly, ktoré by ich upravovali. \cite{MDN}
\end{itemize}

% CSS
\section{CSS}
CSS (Cascading Style Sheets) je deklaratívny jazyk, ktorý dokáže kontrolovať ako sa webový stránky zobrazujú vo webových prehliadačoch. Prehliadače aplikujú CSS štýly priamo na elementy nimi upravené a potom ich zobrazia. Deklarácie štýlov obsahujú \emph{vlastnosti} a~ich \emph{hodnoty}, ktoré určujú ako má webová stránka vyzerať. \cite{MDN} \\

\noindent CSS je možné pridať do HTML dokumentu tromi spôsobmi: 
\begin{itemize}
	\item Importovaním externého CSS súboru v~hlavičke dokumentu.
	\item Vložením medzi element \texttt{<style>} do hlavičky dokumentu.
	\item Vložením jednotlivých vlastností a ich hodnôt do značiek jednotlivých HTML elementov cez parameter \texttt{style}.
\end{itemize}

\noindent Jednotlivé vlastnosti sa elementom priraďujú použitím \emph{CSS selektorov}. Existujú aj selektory alebo kombinátory, ktoré umožňujú zvoliť rodičovské alebo vedľajšie elementy. \cite{MDN} \\

% JavaScript
\section{JavaScript}
JavaScript je populárny \emph{interpretovaný} programovací jazyk. Napriek tomu, že je známy predovšetkým ako skriptovací jazyk pre webové aplikácie, dnes je využívaný mnohými prostrediami mimo webových prehliadačov, ako napríklad Node.js [\ref{subsection:nodejs}] pre tvorbu sieťových aplikácií. \cite{MDN} \\

\noindent Štandardom pre JavaScript je ECMAScript\footnote{\href{https://www.ecma-international.org/}{https://www.ecma-international.org/}}. Od roku 2012 všetky moderné webové prehliadače podporujú ECMAScript verzie 5.1. Staršie prehliadače podporujú aspoň ECMAScript~3. V~roku 2015 bola vydaná verzia ECMAScript 2015 (známa aj ako ECMAScript 6 alebo ES6). Odvtedy je štandard ECMAScript na cykle ročných vydaní. \cite{MDN}

% TypeScript
\section{TypeScript}
\label{theory:typescript}
TypeScript je rozšírenie programovacieho jazyka JavaScript. Jedná sa o~silne typovaný, objektovo orientovaný a kompilovaný programovací jazyk. \cite{TSWeb}

TypeScript je obvykle vhodné skompilovať do natívneho JavaScriptu pre zachovanie kompatibility a lepšiu optimalizáciu. Využívanie TypeScriptu nie je nutné, avšak vďaka vlastnosti silného typovania umožní vývojárovi predísť chybám ešte pred kompiláciou.

\blockquote[Dokumentácia TypeScript \cite{TSWeb}]{Vzťah medzi TypeScriptom a JavaScriptom je unikátny medzi modernými programovacími jazykmi. TypeScript existuje ako vrstva nad JavaScriptom; ponúka vlastnosti JavaScriptu a pridáva svoju vlastnú vrstvu navrch. Táto vrstva je nazývaná \emph{typovací systém TypeScript}.}

\noindent Využitie jednoduchého typu v TypeScripte by mohlo vyzerať následovne: \\

\begin{lstlisting}[language=TypeScript, caption=Príklad zápisu v~programovacom jazyku TypeScript.]
	/* Definicia typu */
	type Person = {
		meno: string;
		priezvisko: string;
	}

	/* Priradenie typu k objektu */
	const Osoba: Person = {
		meno: "Jan";
		priezvisko: "Novak";
	}
\end{lstlisting}

\medskip

\noindent Využitie programovacieho jazyka TypeScript nie je limitované pre vývoj klientských aplikácii. Rovnako ako pri JavaScripte sa jedná o~univerzálny programovací jazyk, ktorý je vďaka nástrojom ako Node.js [\ref{subsection:nodejs}] možné využiť napríklad aj na tvorbu serverových aplikácií.

% React
\section{React}
\label{theory:react}
Populárna JavaScriptová knižnica pre \emph{budovanie uživateľských rozhraní}. React je deklaratívny, efektívny a flexibilný. Dovoľuje vytvárať uživateľské rozhrania zložené z~malých izolovaných častí kódu, nazývaných \emph{komponenty} [\ref{theory:components}]. \cite{React}

\subsection{JSX}
Syntaktické rozšírenie JavaScriptu inšpirované značkovacími jazykmi, avšak s~možnosťou využívať plné schpnosti JavaScriptu. JSX vzniklo z dôvodu, že v~moderných webových aplikáciach bolo čoraz častejšie nutné spájať vykreslovaciu logiku a logiku uživateľských rozhraní. \cite{React} \\

\begin{lstlisting}[language=TypeScript, caption=Príklad využitia JSX v~React aplikácií. \cite{React}]
	const meno = "Jan Novak";
	const element = <h1>Ahoj, {meno}</h1>;

	ReactDOM.render(
		element,
		document.getElementById("root")
	);
\end{lstlisting}

\medskip

\noindent Atribúty JSX tagov môžu prijímať textové reťazce (\texttt{<div className='block'>}) alebo JavaScriptové výrazy (\texttt{<img src=\{item.image\} />}), ktoré sa neskôr vyhodnotia. \\

\noindent Elementy JSX sú kompilované do volaní \texttt{React.createElement()}, ktoré vrátia obyčajné JavaScriptové objekty nazvané \uv{React elementy}. \cite{React} \\

\begin{lstlisting}[language=TypeScript, caption=Príklad jednoduchého React elementu po kompilácií. \cite{React}]
	const element = {
		type: 'h1',
		props: {
			className: 'pozdrav',
			children: 'Ahoj svet!'
		}
	};
\end{lstlisting}

\medskip

\noindent Tieto objekty slúžia Reactu ako \uv{popis} pre zobrazenie. Využíva ich pre zostavenie a urdržovanie aktuálnosti DOM\footnote{DOM -- document object model}. \cite{React}

\subsection{Komponenty}
\label{theory:components}
Komponenty umožňujú vývojárovi rozdeliť uživateľské rozhranie do samostatných, izolovaných a \emph{znovu použiteľných} častí. \cite{React} \\

\noindent Komponenty sa delia na dve skupiny -- \emph{funkcionálne} a \emph{triedne}. Najjednoduchší spôsob ako definovať komponent je obyčajná JavaScriptová funkcia: \\

\begin{lstlisting}[language=TypeScript, caption=Príklad definície funkcionálneho komponentu.]
	const Titulok: React.FC = (props) => {
		return <h1>Vitajte na {props.nazov}</h1>;
	}
\end{lstlisting}

\medskip

\noindent Avšak pre definíciu komponentu možeme použiť aj ES6 triedu: \\

\begin{lstlisting}[language=TypeScript, caption=Príklad definície triedneho komponentu.]
	class Titulok extends React.Component {
		render {
			return <h1>Vitajte na {props.nazov}</h1>;
		};
	}
\end{lstlisting}

\medskip

\blockquote[Dokumentácia React \cite{React}]{Konceptuálne sú komponenty ako obyčajné JavaScriptové funkcie. Prijímajú vstupy nazývané \emph{props} a vracajú React elementy popisujúce zobrazenie na obrazovke.}

\noindent Vďaka veľkej komunite je React jedným z najpoužívanejších nástrojov pre tvorbu užívateľských rozhraní. 

% Next.js
\section{Next.js}
\label{theory:nextjs}
Najväčším problémom moderných React aplikácií je, že \emph{vykresľovanie obsahu prebieha na strane klienta} (client side render). Logika stránok, získavanie dát, smerovanie, všetko prebieha priamo vo webovom prehliadači. To spôsobuje nielen vysoké nároky na výpočetný výkon klientskej stanice, ale aj problémy so SEO optimalizáciou (roboty, ktoré používajú webové vyhľadávacie nástroje ako napríklad Google majú problémy s indexovaním stránok, ktoré sú vykresľované na strane klienta). \\

\noindent Next.js ponúka riešenie na tieto problémy. Umožňuje webové aplikácie vykresliť vopred dvoma spôsobmi:

\begin{itemize}
	\item \texttt{Statické vykresľenie} -- Stránky, sú zobrazované rovnako pri každej požiadavke a nevyžadujú teda akýkoľvek kontext môžu byť vykreslené už pri \emph{komplikácií aplikácie}. HTML štruktúra je teda vygenerovaná vopred a zasielaná klientom, ktorí si ju danú stránku vyžiadali.
	\item \texttt{Vykresľovanie na strane servera} -- Stránky, ktoré vyžadujú kontext pri každej požiadavke (napríklad profilová stránka užívateľa vyžaduje jeho ID pre získanie dát), sú vykreslené vopred len čiastočne a zbytok je prenechaný pre klienta.
\end{itemize}

\noindent V oboch prípadoch je ku HTMl priložený minimálny JavaScript kód, ktorý sa vo webovom prehliadači inicializuje a zaistí tak interaktivitu. Tento proces sa nazýva \emph{hydratácia}. \cite{NextJS} \\

\noindent Ďaľšími benefitmi Next.js sú:
\begin{itemize}
	\item \texttt{Nulová konfigurácia} -- Žiadna vstupná konfigurácia nie je potrebná. Všetky funkcionality Next.js sú dostupné ihneď po inštalácií. \cite{NextJS}
	\item \texttt{Optimalizácia} -- Automatické optimalizovanie balíčkov, ktoré sú vyžadované, čo umoňuje rýchlejšiu kompiláciu. \cite{NextJS}
\end{itemize}

% Vývoj serverových aplikácii
%---------------------------------------------------------------------------
\chapter{Vývoj serverových aplikácii}
\label{theory:server_dev}
\texttt{Server} -- centrálny počítač z~ktorého ostatné počítače získavajú informácie. \cite{CamDict} \\

\noindent Serverová aplikácia je proces spustený na centrálne dostupnom zariadení. Obvykle slúži ako zdroj informácií pre ostatné zariadenia, typicky v~počítačovej sieti. Tento proces očakáva požiadavky a odpovedá na ne vopred určenou reakciou.

Serverové aplikácie je možné implementovať v~rôznych programovacích jazykoch, no medzi najpoužívanejšie patrí PHP, Java, Python alebo JavaScript (popr. rozšírenie TypeScript). Aplikácia má obvykle určený komunikačný protokol, pomocou ktorého prijíma požiadavky a odosiela odpovede.

Táto kapitola pojednáva o základných princípoch vývoja serverových aplikácií. Sekcia \ref{theory:HTTP} je venovaná komunikačnému protokolu HTTP, sekcia \ref{theory:nodejs} JavaScriptovému prostrediu pre tvorbu sieťových aplikcií Node.js, sekcia \ref{theory:databases} relačným databázam, sekcia \ref{theory:graphql} špecifikácií GraphQL. Posledná sekcia \ref{theory:slack_api} približuje otvorenú API aplikácie Slack, ktorej znalosť je nutná pre vytvorenie tejto práce.

% HTTP
\section{HTTP}
\label{theory:HTTP}
Protokol HTTP je využívaný ako generický protokol pre prenos dát (správ) medzi klientom a serverom, napríklad HTML dokumentov. Komunikácia je zahájená klientom, typicky webovým prehliadačom.

\blockquote[RFC 2616 \cite{RFC_HTTP}]{Hypertext Transfer Protocol (HTTP) je protokol na aplikačnej úrovni pre distribuované a spolupracujúce informačné systémy. Protokol HTTP je využívaný iniciatívou World Wide Web od roku 1990.}
 
\subsection{Terminológia}

\begin{itemize}
	\item \texttt{Požiadavka (Request)} -- HTTP správa odoslaná klientom, adresovaná pre server.
	\item \texttt{Odpoveď (Response)} -- HTTP správa odoslaná serverom, adresovaná pre klienta. Odosiela sa po prijatí a spracovaní HTTP požiadavky od klienta.
	\item \texttt{Metóda (Method)} -- Pole v~hlavičke HTTP požiadavky. Definuje operáciu, ktorú má server vykonať po prijatí danej HTTP požiadavky.
\end{itemize}

\subsection{Metódy}
\begin{itemize}
	\item \texttt{OPTIONS} -- Reprezentuje požiadavku na popis komunikačných schopností príjemcu. Umožňuje klientovi vopred zistiť komunikačné možnosti servera bez nutnosti odosielania konkrétnej HTTP požiadavky na daný zdroj dát. \cite{MDN}
	\item \texttt{GET} -- Vyžiadanie si konkrétnych dát. Požiadavky s~metódou GET by dáta mali výhradne získavať, nie ich odosielať. \cite{MDN}
	\item \texttt{POST} -- Odoslanie dát na server. Dáta sú priložené v~tele správy. Typ odosielaných dát určuje hlavička \emph{Content-Type}. \cite{MDN}
	\item \texttt{PUT} -- Vytvorenie nových alebo úprava existujúcich dát pomocou dát priložených v~tele správy. \cite{MDN}
	\item \texttt{DELETE} -- Odstránenie dát. \cite{MDN}
\end{itemize}

\noindent Typy metód, ktoré nesúvisia s~touto prácou boli zámerne neuvedené. \\

\begin{lstlisting}[language=TypeScript, caption=Príklad odoslania HTTP POST metódy v~prostredí TypeScript.]
	fetch("http://localhost:5000/users", {
		method: "POST",
		headers: {
			"Content-Type": "application/json",
		},
		body: JSON.STRINGIFY({
			firstName: "John",
			secondName: "Doe",
		})
	});
\end{lstlisting}

% Node.js
\section{Node.js}
\label{theory:nodejs}
Asynchrónny, udalosťami riadený Javascriptový runtime\footnote{runtime -- behové prostredie programu} vytvorený pre budovanie škálovateľných sieťových aplikácii. \cite{NodeJS} \\

\noindent Node.js umožňuje spustiť JavaScriptový kód mimo prostredia webových prehliadačov a poskytuje rozšírené funkcionality ako prístup k~súborovému systému alebo kryptografické metódy.

\subsection{Node Package Manager - NPM}
Najväčší softvérový register na svete. NPM umožňuje vývojárom open-source softvéru zdieľať JavaScriptové balíčky verejne alebo si vytvoriť súkromný repozitár a balíčky zdieľať iba v~organizácií. \cite{NPM} \\

\noindent NPM je rozdelené do troch hlavných častí:
\begin{itemize}
	\item \texttt{\href{https://npmjs.com/}{Webová stránka}} -- Vyhľadávanie balíčkov, správu profilov a organizácií. \cite{NPM}
	\item \texttt{Command Line Interface (CLI)} -- Konzolové rozhranie pre interakciu s~NPM. \cite{NPM}
	\item \texttt{Register} -- Verejná databáza JavaScriptového softvéru a metadát. \cite{NPM}
\end{itemize}

\noindent Hlavnou časťou je pre vývojára práve \texttt{CLI}, ktoré mu umožňuje spravovať balíčky vo svojej aplikácií. Nainštalované balíčky ale ich závislosti sú inštalované do zložky v~koreňovom adresári projektu \texttt{node\_modules}. Nastavenia, informácie o~projekte a~zoznam nainštalovaných balíčkov sa uchováva v~súbore \texttt{package.json}, taktiež umiestenom v~koreňovom adresári projektu.

% Relačné databázy
\section{Relačné databázy}
\label{theory:databases}
V~dnešnej dobe štandardný typ databázy pre ukladanie a prístup k~dátam vo vzájomných väzbách. Dáta sú rozdelené do tabuliek, kde každý riadok reprezentuje jednu entitu. Entita je vždy identifikovateľná svojim primárnym kľúčom, ktorý musí byť v~danej tabuľke unikátny. Každý stĺpec tabuľky má vopred definovaný dátový typ a nesie hodnoty pre každý element v~tabuľke. \\

\noindent \emph{SQL} (\emph{Structured Query Language}) je štandardizovaný dotazovací jazyk pre prístup a manipuláciu s~dátami v~relačných databázach.

\subsection{PostgreSQL}
Plnohodnotný open-source objektovo-orientovaný databázový systém, ktorý je v~aktívnom vývoji už viac ako 30 rokov. PostgreSQL je kompatibilný so všetkými hlavnými platformami ako Linux, MacOS, Solaris a Windows. Tento systém si obľúbili mnohé veľké spoločnosti po celom svete. \cite{PostgreSQL}

\subsection{Prisma}
Prisma je open-source \emph{sada nástrojov pre správu databázy}. Nahradzuje tradičné ORM\footnote{ORM -- Object-relational mapping} a uľahčuje prístup k~databáze pomocou automaticky generovaného klienta pre zostavovanie dotazov. Prisma je implementovaná v~jazyku TypeScript, čo znamená, že \emph{podporuje striktnú typovú kontrolu} a zmenšuje tým pravdepodobnosť chyby vo vývoji. \cite{Prisma} \\

\begin{lstlisting}[language=TypeScript, caption=Príklad tvorby užívateľa pomocou nástroja Prisma. \cite{Prisma}]
	await prisma.users.create({
			data: {
			firstName: "Alice",
			email: "alice@prisma.io",
			active: true,
		}
	})
\end{lstlisting}

\noindent Vďaka sade nástrojov \texttt{prisma} nie je nutné pristupovať k databáze priamo pomocou SQL dotazov, ale je možné využiť API automaticky generovaného klienta.

% GraphQL
\section{GraphQL}
\label{theory:graphql}
Dotazovací (query) jazyk pre API a runtime na strane servera pre vykonávanie dotazov pomocou \emph{vopred definovaného} typového systému. GraphQL nie je viazané žiadnou špecifickou technológou ukladania dát, naopak pracuje ako backend pre už existujúci kód a~dáta. \cite{GraphQL} GraphQL nie je implementácia, ale iba špecifikácia. Existujú mnohé implementácie GraphQL implementované v~mnohých programovacích jazykoch, na rôznych platformách. \\

\noindent GraphQL služba je vytvorená definovaním typov a ich jednotlivých polí. Každému polu každého typu je následne priradená funkcia. \cite{GraphQL} Tieto funkcie sú nazývané \uv{\emph{resolvers}}. Tieto funkcie vyhodnocujú dáta jednotlivých polí pri prichádzajúcej požiadavke. \\

\begin{lstlisting}[label=lstlisting:graphql_schema, language=GraphQL, caption=Príklad jednoduchej GraphQL schémy. \cite{GraphQL}]
	type Query {
		me: User
	}

	type User {
		id: ID
		name: String
	}
\end{lstlisting}

\medskip

\noindent Po spustení GraphQL služby (typicky URL adresa dostupná cez HTTPS) očakáva GraphQL dotazy, ktoré spracováva a vyhodnocuje. Obdržané dotazy najskôr skontroluje, či požadujú iba typy a polia, ktoré sú definované. V~prípade, že sú dotazy v~poriadku, Graphql spustí požadované resolvery a vracia výsledok. \cite{GraphQL} \\

\noindent GraphQL rozlišuje tri typy operácií s~dátami. Prvý typ operácie je \emph{query} a slúži výhradne na čítanie dát. Druhým typ operácie sa nazýva \emph{mutation} a slúži na úpravu, vkladanie alebo odstránenie dát. Tretím a posledným typom operácie je \emph{subscription}. Ak GraphQL služba spracuje operáciu typu \emph{subscription}, pošle vyžiadané dáta klientovi pri každej ich zmene. \\

\noindent Pre ukážku, \emph{query} dotaz na GraphQL službu so schémou ukázanou vo výpise \ref{lstlisting:graphql_schema}: \\

\begin{lstlisting}[caption=Príklad \emph{query} dotazu na GraphQL službu. \cite{GraphQL}]
	query {
		me {
			name
		}
	}
\end{lstlisting}

\medskip

\ldots by po spracovaní GraphQL službou mohol vrátiť JSON\footnote{JSON - javascript object notation} objekt: \\

\begin{lstlisting}[caption=Príklad dát obdržaných z~GraphQL služby. \cite{GraphQL}]
	{
		"me": {
			"name": "Luke Skywalker"
		}
	}
\end{lstlisting}

\subsection{Apollo Client}
Implementácia GraphQL špecifikácie pre klientské aplikácie. Apollo Client je priamo prepojený s~knižnicou React a poskytuje nástroje umožňujúce rýchly vývoj webových aplikácií, ktoré získavajú dáta z~GraphQL služby. \\

\blockquote[Dokumentácia Apollo GraphQL \cite{Apollo}]{Apollo Client je knižnica umožňujúca plnohodnotnú správu stavu dát v~JavaScriptových aplikáciach. Umožňuje vývojárovi len napísať požadovaný GraphQL dotaz a Apollo Client sa už postará o~získanie, uloženie dát do cache a obnovenie uživateľského rozhrania. Využívanie knižnice Apollo Client vedie vývojára ku správnej štruktúre kódu.}

\noindent Apollo GraphQL umožňuje využiť svoju cache aj pre ukladanie lokálneho stavu aplikácie. Vďaka tomu sa pre danú aplikáciu stáva jediným zdrojom dát.

\subsection{Apollo Server}
Serverová časť GraphQL implementácie, kompatibilná so všetkými GraphQL kientami, vrátane Apollo Client. Apollo Server dokáže pracovať ako samostatný GraphQL server alebo ako súčasť už existujúcej aplikácie ako napríklad Express\footnote{\href{https://expressjs.com/}{https://expressjs.com/}} server. \cite{Apollo} \\

\begin{figure}[h]
	\centering
	\includegraphics[scale=0.8]{obrazky-figures/apollo_server_diagram}
	\caption{Diagram architektúry Apollo Server. \cite{Apollo}}
\end{figure}

\noindent Apollo Server dokáže pôsobiť ako \emph{jednotná brána} pre všetky klientské aplikácie. Brána má predom definovanú schému, tzn. že klient presne vie aké operácie môže vykonávať a aké dáta môže očakávať späť.

% Slack API
\section{Slack API}
\label{theory:slack_api}
Aplikácia pre tímovú a pracovnú komunikáciu Slack umožňuje vývojárom vytvoriť vlastné aplikácie a prepojiť ich so Slackom pomocou ich \emph{otvorenej API}. \cite{SlackAPI} \\

\noindent Aplikácie môžu vykonávať niektoré akcie bežných užívateľov:

\begin{itemize}
	\item \emph{Odosielať správy} do rôznych koverzácií. \cite{SlackAPI}
	\item \emph{Čítať správy a konverzácie}. \cite{SlackAPI}
	\item \emph{Vytvoriť, archivovať a spravovať konverzácie}. \cite{SlackAPI}
	\item \emph{Reagovať na označenia} od užívateľov. \cite{SlackAPI}
\end{itemize}

\ldots no API aplikáciam umožňuje aj akcie mimo uživateľských práv:

\begin{itemize}
	\item \emph{Otvoriť dialógové okná} pre získanie alebo zobrazenie extra informácií. \cite{SlackAPI}
	\item \emph{Zostaviť interaktívne komponenty} na ktoré môzu reagovať \cite{SlackAPI}
	\item \emph{Vytvoriť a aktualizovať \uv{Home tab} (domovskú obrazovku)} kde môže užívateľ interagovať s~aplikáciou. \cite{SlackAPI}
	\item \emph{Definovať skratky} vďaka ktorým užívateľ môže rýchlo vykonať akcie v~aplikácií. \cite{SlackAPI}
\end{itemize}

\subsection{Block Kit}
Framework pre tvorbu uživateľského rozhrania pre Slack aplikácie, ktorý ponúka kompromis kontroly a flexibility pri budovaní rozhraní v~správach. \cite{SlackAPI}  Uživateľské rozhranie sa skladá pomocou blokov, ktoré sú vo forme JSON objekov prenášané do Slack aplikácie. Bloky sa delia na dve hlavné kategórie -- \emph{interactive} a \emph{layout} bloky. \\

\begin{lstlisting}[caption=Príklad jednoduchého bloku v~Slack aplikácií.]
	{
		"type": "section",
		"text": {
			"type": "mrkdwn",
			"text": "Hello, *World*!"
		}
	},
\end{lstlisting}

\medskip

\noindent Bloky sa vkladajú do lineárneho zoznamu nazvanom \emph{view}, ktorý je následne odoslaný na príslušnú časť API, aby bol spracovaný. Niektoré bloky podporujú formátovanie pomocou modifikovanej verzie jazyka \emph{markdown}.

% Headless CMS
%---------------------------------------------------------------------------
\chapter{Headless CMS}
\label{theory:headless}
Headless CMS štandardne disponujú rozhraním REST\footnote{REST -- Representational State Transfer} alebo GraphQL, ktoré implementuje aj táto práca. Headless redakčný systém \emph{nerieši zobrazenie samotného obsahu}. Jediný spôsob ako obsah získať je využiť niektoré z~dostupných rozhraní poskytované konkrétnym riešením. Výhodou oproti tradičným redakčným systémom je možnosť získané dáta optimálne zobraziť na rôznych zariadeniach. 

\begin{figure}[h]
	\centering
	\includegraphics{obrazky-figures/headless_cms_graph.pdf}
	\caption{Ilustračná schéma generického headless redakčného systému.}
\end{figure}

\noindent Členenie obsahu v~takýchto redakčných systémoch je typicky v~dvoch vrstvách -- \emph{kategórie} a \emph{obsahové typy} (komponenty). \\

\noindent \emph{Kategórie} sú zoznamy združujúce jednotlivé komponenty, môžu byť homogénne (všetky prvky zoznamu sú jedného typu) alebo heterogénne (prvky zoznamu sú typicky iných typov). \\

\noindent \emph{Komponenty} sú atomickými prvkami headless redakčných systémov. Môžu nadobúdať rôznych typov, ktoré určujú ich vnútornú dátovú štruktúru. Typický príklad často používaných typov komponentov je napríklad \texttt{prostý text} alebo \texttt{odkaz}. Niektoré headless redakčné systémy umožňujú vytvárať aj vlastné typy komponentov a tak si prospôsobiť dáta vlastným špecifickým potrebám.

% Strapi
\section{Strapi}
Najpopulárnejší Headless redakčný systém. Disponuje administračným panelom zostaveným na mieru, REST aj GraphQL rozhraním, systémom uživateľských práv a mnohými inými vlastnosťami. Tento redakčný systém má aj obchod s~aplikáciami, ktoré si môžu používatelia pridať a tým rozšíriť funkcionalitu. 

\blockquote[Dokumentácia Strapi.io \cite{StrapiDocs}]{Strapi je flexibilný, open-source\footnote{open-source -- otvorený kód, zväčša vyvíjaný komunitou} Headless CMS\footnote{CMS -- ang. content management system (redakčný systém)}, ktorý dáva vývojárom slobodu voľby ich obľúbených nástrojov a zároveň dovoľuje editorom jednoducho spravovať a distribuovať ich obsah.}

\noindent V~prípade, že by užívateľ Strapi mal nakonfigurovanú kolekciu s~názvom \uv{restaurants}, získanie počtu týchto kolekcií by mohlo vyzerať takto: \\

\begin{lstlisting}[caption=Príklad HTTP požiadavku na REST rozhranie Strapi.]
	GET "http://localhost:1337/restaurants/count" // Odpoved: 1
\end{lstlisting}

\medskip

\noindent Pre použitie Strapi je nutné systém spustiť na vlastnej infraštruktúre, pripojiť k~predom vytvorenej relačnej databáze a celý systém nakonfigurovať.

% Netlify CMS
\section{Netlify CMS}
Netlify na rozdiel od väčšiny headless redakčných systémov nevyužíva pre ukladanie svojich dát relačnú databázu. Pre uloženie celého obsahu webovej aplikácie využíva repozitáre vytvorené v~prostredí \texttt{git}.

\blockquote[Dokumentácia Netlify CMS \cite{NetlifyDocs}]{Jadro Netlify CMS tvorí React [\ref{theory:react}] aplikácia ktorá využíva rozhranie pre prácu s~GitHub\footnote{\href{https://developer.github.com/v3/}{https://developer.github.com/v3/}}, GitLab\footnote{\href{https://docs.gitlab.com/ee/api/}{https://docs.gitlab.com/ee/api/}} alebo Bitbucket\footnote{\href{https://confluence.atlassian.com/bitbucket/}{https://confluence.atlassian.com/bitbucket/}} API.}

\noindent Netlify sa využíva väčšinou pre menšie stránky ako sú napríklad dokumentácie alebo produktové stránky, pretože umožňuje udržovať obsah relevantný k~danej verzií produktu.

% Návrh riešenia
%---------------------------------------------------------------------------
\chapter{Návrh riešenia}
Kapitola popisuje jednotlivé etapy vzniku návrhu služby, ktorá je predmetom tejto práce (ďalej ako \uv{\textbf{Slackify}}). Výsledný návrh vznikol po nadobudnutí požadovaných znalostí a analýz požiadavkov kladených na výsledný produkt.

Sekcia \ref{design:assignment} rozoberá zadanie práce, sekcia \ref{design:architecture} predstavuje architektúru systému aj s neúspešnými iteráciami, sekcia \ref{theory:data_types} popisuje špeciálne dátové typy navrhovaného systému. Posledná sekcia \ref{design:use_case} je venovaná konkrétnym prípadom užitia Slackify.

% Zadanie práce
\section{Zadanie práce}
\label{design:assignment}
Zadanie, ktorého celé znenie je priložené na \hyperlink{page.2}{strane 2}, popisuje výsledné riešenie práce ako \emph{headless} redakčný systém. Súhrn hlavných požiadaviek na systém je nasledovný:

\begin{itemize}
	\item Užívateľ systému by mal byť schopný spravovať obsah v~\emph{plnej miere} a bez obmedzení v~prostredí aplikácie Slack.
	\item Užívateľ by mal byť schopný začať využívať službu \emph{okamžite} po nainštalovaní do svojho pracovného prostredia v~aplikácií Slack. To znamená, že žiadna konfigurácia alebo nasadenie na vlastnú infraštruktúru by nemalo byť požadované.
	\item Všetky vytvorené dáta by mali byť dostupné cez verejné API.
	\item Systém by mal disponovať webovým rozhraním, odkiaľ by malo byť možné spravovať obsah a nastavenia.
\end{itemize}

\noindent Hlavnou výsadou práce nie je čo najväčšia flexibilita, ale jednoduchosť v~použití a možnosť rýchleho nasadenia. 

\subsection{Odchýlky návrhu od pôvodného zadania}
Slack API umožňuje vytvoriť uživateľské rozhranie zložené z~maximálneho počtu 100 blokov. Táto skutočnosť znamená, že nie je možné zobrazovať napríklad dlhé zoznamy prvkov. Pre tento dôvod je maximálny počet zobrazených prvkov v~zoznamoch Slack aplikácie limitovaný. Prvky zoznamov nad určený limit sú skryté a užívateľ ich môže spravovať výhradne vo webovom prostredí. 

% Návrh architektúry
\section{Návrh architektúry}
\label{design:architecture}
Na začiatkoch navrhovania architektúry bola služba rozdelená do dvoch hlavných častí podľa vzoru \emph{backend} a \emph{frontend}. Frontend bola webová služba a backend poskytovala GraphQL bránu pre prístup k~dátam a logike pre frontend, ale slúžila aj ako prístupový bod pre koncových užívateľov headless redakčného systému a tiež ako bod pre komunikáciu so Slack API. Dáta boli ukladané v~relačnej databáze. \\

\noindent Aj keď v~prvopočiatkoch tvorby návrhu sa táto architekúra zdala ako vyhovujúca, neskôr sa ukázalo ako lepšia možnosť nasledujúce rozdelenie do štyroch menších služieb:

\begin{itemize}
	\item \texttt{frontend} -- Webová konzola poskytujúca prístup k~obsahu a nastaveniam headless redakčného systému. Táto časť zostala nezmenená.
	\item \texttt{service-slack} -- Serverová aplikácia zodpovedná za komunikáciu so Slack API. Udalosti a akcie prichádzajúce zo Slack aplikácie smerujú na túto službu, kde sú spracované a vyhodnotené.
	\item \texttt{service-private} -- Serverová GraphQL aplikácia poskytujúca logiku a prístup k~dátam webovej konzole (\texttt{frontend} službe).
	\item \texttt{service-public} -- Serverová GraphQL aplikácia poskytujúca prístup k~obsahu redakčného systému (verejné API headless redakčného systému).
	\item \texttt{database} -- Relačná databáza obsahujúca uživateľské dáta, nastavenia a obsah od koncových užívateľov.
\end{itemize}

\noindent Takáto architektúra sa oproti jednej veľkej backend službe ukázala ako výhodná z~dvoch hlavných dôvodov. Prvým dôvodom je \emph{omnoho jednoduchšia údržba} viacerých menších služieb (napríklad udržovanie logickej súborovej štruktúry bolo pri jednej monolitickej službe obtiažne). Druhým významným dôvodom je možnosť jednotlivé služby \emph{verziovať a vydávať samostatne}, pretože fungujú nezávisle jedna na druhej.


\begin{figure}[h]
	\centering
	\includegraphics[scale=1.4]{obrazky-figures/architecture}
	\caption{Vizualizácia architektúry podľa návrhu.}
\end{figure}

Jednotlivé služby sú ešte rozdelené do dvoch skupín. \texttt{Frontend} a \texttt{service-public} sú služby dostupné pre verejnosť (\emph{public services}), \texttt{service-private} a \texttt{service-slack} sú pred verejnosťou skryté (\emph{private services}). 

% Dátové typy
\section{Dátové typy}
\label{theory:data_types}
Slackify je headless redakčný systém obsahujúci dva hlavné dátové typy -- \emph{collection} (kolekcia) a \emph{component} (komponent). Pomocou týchto dvoch dátových typov je možné vytvoriť logickú štruktúru obsahu. U oboch dátových typov sa uchováva momentálny stav uverejnenosti \emph{published}, repretenzovaný dátovým typom boolean.

\subsection{Component}
Component (komponent) je základná dátová struktúra v Slackify. Celý obsah redakčného systému je zložený z komponentov usporiadaných do kolekcií [\ref{design:collection}]. Každý komponent má povinnú položku \emph{type} (typ), ktorá určuje ďaľšie zloženie štruktúry.

\subsubsection{Typ \uv{\texttt{Plain Text}}}
Najjednoduchší typ obsahujúci jediné povinné textové pole s názvom \emph{text}.

\subsubsection{Typ \uv{\texttt{Article}}}
Typ pre vytvorenie jednoduchého článku. Obsahuje povinné textové položky \emph{title} (titulok), \emph{content} (obsah článku) a nepovinnú textovú položku \emph{lead} (úvodný text).

\subsubsection{Typ \uv{\texttt{Link}}}
Typ pre vytvorenie hypertextového odkazu. Obsahuje povinnú textovú položku \emph{url} (cieľová adresa) a voliteľnú textovú položku \emph{text}.

\subsection{Collection}
\label{design:collection}
Collection (kolekcia) je zoznam slúžiaci pre kategorizovanie jednotlivých komponentov. Každá kolekcia ich môže obsahovať 0 až $n$. Jedná sa o \emph{homogénny zoznam}, tzn. že pri tvorbe kolekcie sa vždy musí zvoliť práve jeden typ komponentov z ktorých sa môže zoznam skladať. Každá kolekcia má povinnú položku \emph{name} (názov) a nepovinnú \emph{description} (popis).

% Diagram prípadov použitia
\section{Diagram prípadov použitia}
\label{design:use_case}
Priložené diagramy prípadov použitia zobrazujú chovanie systému z~pohľadu registrovaných užívateľov a zariadenia, ktoré sa snaží získať obsah z headless CMS.

\subsection{Viewer}
Každému novému užívateľovi, ktorý prepojí svoj účet v aplikácií Slack so Slackify je automaticky priradená základná užívateľská rola \emph{Viewer}. \\

\noindent Užívateľ s rolou \emph{Viewer} môže vykonávať nasledujúce akcie:

\begin{itemize}
	\item \texttt{Získať detail kolekcie/komponentu} -- Zobrazenie všetkých informácií o kolekcií alebo alebo komponente. Dáta je možné iba čítať, nie modifikovať.
	\item \texttt{Získať list kolekcií/komponentov} -- Zobrazenie zoznamu kolekcií alebo komponentov. Komponenty je možné zobraziť všetky alebo podľa pridelených kolekcií. Dáta je možné iba čítať, nie modifikovať.
\end{itemize}

\begin{figure}[h]
	\centering
	\includegraphics[scale=0.9]{obrazky-figures/viewer_use_case}
	\caption{Diagram prípadov použitia -- \emph{Viewer}.}
\end{figure}

\subsection{Author}
Užívateľská rola \emph{Author} zahŕňa všetky právomoci, ktoré mala rola \emph{Viewer} a rozširuje ich o možnosť vykonávať nasledujúce akcie:

\begin{itemize}
	\item \texttt{Vytvoriť kolekciu/komponentu} -- Po zadaní a potvrdení vstupných údajov je nová kolekcia alebo komponent pridaná do databázy. Nové komponenty sú inicializované so stavom \emph{unpublished} (nepublikovaná).
	\item \texttt{Upraviť kolekciu/komponentu} -- Kolekcie alebo komponety je možné upraviť v~ľubovoľný čas. Jediná hodnota, ktorú nie je možné modifikovať je typ kolekcie alebo komponentu.
\end{itemize}

\begin{figure}[h]
	\centering
	\includegraphics[scale=0.9]{obrazky-figures/author_use_case}
	\caption{Diagram prípadov použitia -- \emph{Autor}.}
\end{figure}

\subsection{Editor}
Užívateľská rola \emph{Editor} zahŕňa všetky právomoci, ktoré mala rola \emph{Author} a rozširuje ich o možnosť vykonávať nasledujúce akcie:

\begin{itemize}
	\item \texttt{Zmazať kolekciu/komponent} -- Kolekcie alebo komponety je možné zmazať v~ľubovoľný čas. Pri zmazaní kolekcie sa zmazú aj všetky komponenty k~nej priradené.
	\item \texttt{Publikovať kolekciu/komponentu} -- Publikovaná kolekcia alebo komponent je okamžite dostupná verejne. \emph{Editor} rovnako môže kolekcie alebo komponenty skryť.
\end{itemize}

\begin{figure}[h]
	\centering
	\includegraphics[scale=0.9]{obrazky-figures/editor_use_case}
	\caption{Diagram prípadov použitia -- \emph{Editor}.}
\end{figure}

\subsection{Owner}
Užívateľská rola \emph{Owner} je rola s najvyššími právomocami. Zahŕňa všetky právomoci, ktoré mala rola \emph{Editor} a rozširuje ich o možnosť vykonávať nasledujúce akcie:

\begin{itemize}
	\item \texttt{Zmeniť rolu užívateľa} -- Umožňuje modifikovať užívateľské role ostatných užívateľov. Užívateľ nemôže zmeniť rolu sám sebe. 
\end{itemize}

\begin{figure}[h]
	\centering
	\includegraphics[scale=0.9]{obrazky-figures/owner_use_case}
	\caption{Diagram prípadov použitia -- \emph{Owner}.}
\end{figure}

\subsection{Zariadenie}
Obsah zo Slackify je možné získať z~verejného API (služba \texttt{service-public}) prostredníctvom akéhokoľvek zariadenia, ktoré dokáže komunikovať cez protokol HTTP. Pre ukážku diagramu prípadov použitia je na mobilnom telefóne nainštalovaná aplikácia, ktorá komunikuje s~rozhraním Slackify. \\

\noindent \emph{Zariadenie} môže v~prostredí Slackify vykonávať nasledujúce akcie:

\begin{itemize}
	\item \texttt{Vyžiadať detail kolekcie/komponentu} -- Informácie o jednej kolekcií alebo komponente je možné vyžiadať od API pomocou ich unikátneho ID.
	\item \texttt{Vyžiadať list kolekcií} -- Zariadenie si môže vyžiadať zoznam kolekcií. Zoznam kolekcií je v základnom nastavení zoradený chronologicky podľa času vytvorenia. Výsledky je možné zoradiť alebo filtrovať pomocou parametrov operácie.
	\item \texttt{Vyžiadať list komponentov} -- Pomocou unikátneho ID kolekcie je možné vyžiadať k nej priradené komponenty. V prípade neuvedenia ID kolekcie je navrátený lisť všetkých komponentov v danom Slack pracovnom prostredí (zoradené chronologicky podľa času vytvorenia). Výsledky je možné zoradiť alebo filtrovať pomocou parametrov operácie.
\end{itemize}

\begin{figure}[h]
	\centering
	\includegraphics[scale=0.9]{obrazky-figures/device_use_case}
	\caption{Diagram prípadov použitia -- \emph{Zariadenie}.}
\end{figure}

% Implementácia databázovej vrstvy
%---------------------------------------------------------------------------
\chapter{Implementácia databázovej vrstvy}
Služby \texttt{service-private}, \texttt{service-public} a \texttt{service-slack} zdieľajú jednu spoločnú databázovú vrstvu. Pre zostavenie databázy a prístup k nej je využitá sada nástrojov \texttt{prisma}, bližšie popísaná v sekcií \ref{theory:databases}.

% Schéma
\section{Schéma}
Sada nástrojov \texttt{prisma} vyžaduje hlavný konfiguračný súbor pomocou ktorého sa dokáže pripojiť k databáze, obvykle nazvaný \texttt{schema.prisma}. Tento súbor obsahuje tri hlavné časti:

\begin{itemize}
	\item \texttt{datasource} -- Špecifikuje typ zdroja dát (databázy) a údaje potrebné k nadviazaniu úspešného pripojenia.
	\item \texttt{generator} -- Nastavenia databázového klienta, ktorý je automaticky generovaný podľa dátového modelu.
	\item \texttt{model} -- Jednotlivé dátové modely, bližšie popísané v sekcií \ref{}.
\end{itemize}

\noindent Pomocou tohto konfiguračného súboru dokáže \texttt{prisma} zostaviť databázu a automaticky vygenerovať databázového klienta. \\

% TODO: Add language definition for prisma files
\begin{lstlisting}[label={impl:code:prisma_schema}, caption=Špecifikácia \texttt{datasource} a \texttt{generator} v konfiguračnom súbore \texttt{prisma}.]
	datasource db {
		provider = "postgresql"
		url      = env("DATABASE_URL")
		enabled  = env("DATABASE_URL")
	}

	generator cient_common {
		provider = "prisma-client-js"
		output   = "../../node_modules/@prisma/client"
	}
\end{lstlisting}

\medskip

\noindent Vo výpise \ref{impl:code:prisma_schema} je ukázaná časť konfiguračného súboru v použitého v Slackify. Definuje PostgreSQL databázu ako typ zdroja dát (\texttt{datasource}) a špecifikuje URL pre pripojenie pomocou environmentálnej premennej \texttt{DATABASE\_URL}. Parameter \uv{\emph{enabled}} hovorí, že daný zdroj dát je povolený, ak je premenná \texttt{DATABASE\_URL} definovaná.

% Dátové modely
\section{Dátové modely}
Dátové modely definované v konfiguračnom súbore \texttt{prisma} reprezentujú entity využívané v aplikácií. Sada nástrojov \texttt{prisma} tieto modely automaticky mapuje do jednotlivých tabuliek v databáze. \\

\noindent Slackify obsahuje dva hlavné modely \texttt{Collection} a \texttt{Component} reprezentujúce dáta uložené v redakčnom systéme a modely \texttt{User} a \texttt{Team} slúžiacie pre uchovávanie informácií o užívateľoch a tímoch \emph{prepojených s aplikáciou Slack}.

\subsection{Model \texttt{Collection}}
Dátový model \texttt{Collection} (kolekcia) slúžiaci ako zoznam pre jednotlivé komponenty. Sú to homogénne zoznamy, ktoré môžu obsahovať iba komponenty jednoho typu.\\

\begin{lstlisting}[caption=Dátový model \texttt{Collection} v konfiguračnom súbore \texttt{prisma}.]
	model Collection {
		id          String        @default(cuid()) @id
		name        String
		type        ComponentType
		published   Boolean       @default(false)
		description String?
		team        Team          @relation(...)
		teamId      String
		components  Component[]
		createdAt   DateTime      @default(now())
		updatedAt   DateTime      @updatedAt
	}
\end{lstlisting}

\medskip

\noindent Popis niektorých významých položiek obsiahutých v dátovom modeli:

\begin{itemize}
	\item \texttt{type} -- Určuje jediný typ komponentu, ktorý je možné vložiť do danej kolekcie. Typ komponentu je bližšie popísaný v sekcií \ref{impl:model:component}.
	\item \texttt{team} -- Relácia s dátovým modelom \texttt{Team} ku ktorému je kolekcia priradená. Každá kolekcia musí byť priradená k práve jednej entite modelu \texttt{Team}.
	\item \texttt{components} -- Relácia 1~:~$n$ všetkých komponentov priradených ku kolekcií. Každá kolekcia môže mať 0 až $\infty$ priradených komponentov.
\end{itemize}

\subsection{Model \texttt{Component}}
\label{impl:model:component}
Dátový model \texttt{Component} (komponent) reprezentujúci malú časť obsahu v Slackify. Každý komponent \emph{musí mať zvolený typ}, ktorý určuje vnútornú štruktúru dát uložených v komponente. \\

\begin{lstlisting}[caption=Dátový model \texttt{Component} v konfiguračnom súbore \texttt{prisma}.]
	model Component {
		id              String                  @default(cuid()) @id
		type            ComponentType           @default(PLAIN_TEXT)
		published       Boolean                 @default(false)
		collection      Collection              @relation(...)
		collectionId    String
		author          User                    @relation(...)
		authorId        String
		team            Team                    @relation(...)
		teamId          String
		plainTextDataId String?
		plainTextData   PlainTextComponentData? @relation(...)
		articleDataId   String?
		articleData     ArticleComponentData?   @relation(...)
		linkDataId      String?
		linkData        LinkComponentData?      @relation(...)
		createdAt       DateTime                @default(now())
		updatedAt       DateTime                @updatedAt
	}
\end{lstlisting}

\medskip

\noindent Popis niektorých významých položiek obsiahutých v dátovom modeli:

\begin{itemize}
	\item \texttt{type} -- Typ komponentu nadobúdajúci jednu z hodnôt výčtového typu \texttt{ComponentType}.
	\item \texttt{collection} -- Relácia s dátovým modelom \texttt{Collection}. Každý komponent musí byť priradený k práve jednej kolekcií.
	\item \texttt{author} -- Relácia s dátovým modelom \texttt{Author}. Každý komponent musí byť priradený k právej jednomu autorovi.
	\item \texttt{team} -- Relácia s dátovým modelom \texttt{Team}. Každý komponent musí byť priradený k práve jednomu tímu.
\end{itemize}

\noindent Hodnota položky \texttt{type} určuje práve jednu z položiek \texttt{plainTextData}, \texttt{articleData} alebo \texttt{linkData} ako povinnú. Tieto položky ukazujú na rôzne špecifické dátové modely, ktoré reprezentujú obsah daných komponentov. Napríklad ak položka \texttt{type} má hodnotu \texttt{LINK}, položka \texttt{linkData} \emph{musí} byť reláciou na dátový model \texttt{LinkComponentData}, ktorý obsahuje povinné pole URL a nepovinné pole text. \\

\begin{lstlisting}[caption=Dátový model \texttt{LinkComponentData} v konfiguračnom súbore \texttt{prisma}.]
	model LinkComponentData {
		id   String  @default(cuid()) @id
		text String?
		url  String
	}
\end{lstlisting}

\subsubsection{Výčtový typ \texttt{ComponentType}}
Výčtový typ (enum), ktorý môže nadobúdať jednu z hodnôt \texttt{PLAIN\_TEXT}, \texttt{ARTICLE} alebo \texttt{LINK}. \\

\begin{lstlisting}[caption=Výčtový typ \texttt{ComponentType} v konfiguračnom súbore \texttt{prisma}.]
	enum ComponentType {
		PLAIN_TEXT
		ARTICLE
		LINK
	}
\end{lstlisting}

\subsection{Model \texttt{User}}
Dátový model \texttt{User} (užívateľ) reprezentujúci informácie o uživateľskom účte prepojenom s aplikáciou Slack. \\

\begin{lstlisting}[caption=Dátový model \texttt{User} v konfiguračnom súbore \texttt{prisma}.]
	model User {
		id          String      @id
		email       String      @unique
		name        String
		role        UserRole    @default(VIEWER)
		accessToken String      @unique
		image_24    String?
		image_32    String?
		image_48    String?
		image_72    String?
		image_192   String?
		image_512   String?
		team        Team        @relation(fields: [teamId], references: [id])
		teamId      String
		components  Component[]
	}
\end{lstlisting}

\medskip

\noindent Popis niektorých významých položiek obsiahutých v dátovom modeli:

\begin{itemize}
	\item \texttt{id} -- Jedinečný identifikátor užívateľa, prevzatý z autentifikačnej API aplikácie Slack.
	\item \texttt{role} -- Užívateľská rola nadobúdajúca jednu z hodnôt výčtového typu \texttt{UserRole}. Určuje právomoci užívateľa, inicializuje sa s hodnotou \texttt{VIEWER}.
	\item \texttt{team} -- Relácia s dátovým modelom \texttt{Team}. Každý užívateľ musí byť priradený k práve jednomu tímu.
	\item \texttt{components} -- Relácia 1~:~$n$ všetkých komponentov, ktorým je uživateľ autorom. Každý užívateľ môže byť autorom 0 až $\infty$ komponentov.
\end{itemize}

\subsubsection{Výčtový typ \texttt{UserRole}}
Výčtový typ (enum), ktorý môže nadobúdať jednu z hodnôt \texttt{VIEWER}, \texttt{AUTHOR}, \texttt{EDITOR} alebo \texttt{OWNER}. \\

\begin{lstlisting}[caption=Výčtový typ \texttt{UserRole} v konfiguračnom súbore \texttt{prisma}.]
	enum UserRole {
		OWNER
		EDITOR
		AUTHOR
		VIEWER
	}
\end{lstlisting}

\subsection{Model \texttt{Team}}
Dátový model \texttt{Team} (tím) reprezentujúci informácie o tíme (pracovnom prostredí, workspace) v aplikácií Slack. \\

\begin{lstlisting}[caption=Dátový model \texttt{Team} v konfiguračnom súbore \texttt{prisma}.]
	model Team {
		id          String       @id
		name        String
		domain      String       @unique
		accessToken String       @unique
		collections Collection[]
		users       User[]
		components  Component[]
	}
\end{lstlisting}

\medskip

\noindent Popis niektorých významých položiek obsiahutých v dátovom modeli:

\begin{itemize}
	\item \texttt{id} -- Jedinečný identifikátor tímu, prevzatý z autentifikačnej API aplikácie Slack.
	\item \texttt{accessToken} -- Generovaný prístupový kód, ktorý je nutné využiť pre získanie obsahu z redakčného systému cez verejnú api (službu \texttt{service-public}). 
	\item \texttt{collections} -- Relácia 1~:~$n$ všetkých kolekcií, ktoré sú vytvorené pod daným tímom. Každý tím môže mať priradených 0 až $\infty$ kolekcií.
	\item \texttt{users} -- Relácia 1~:~$n$ všetkých užívateľov, ktorí sú členmi daného tímu. Každý tím môže mať 0 až $\infty$ užívateľov.
	\item \texttt{components} -- Relácia 1~:~$n$ všetkých komponentov, ktoré sú vytvorené pod daným tímom. Každý tím môže mať priradených 0 až $\infty$ komponentov.
\end{itemize}

% Automaticky generovaný databázový klient
\section{Automaticky generovaný databázový klient}
Pomocou konfiguračného súboru \texttt{prisma} automaticky zostrojí striktne typovaného databázové klienta, ktorý dokáže zostavovať dotazy pre databázu. Pomocou tohto klienta je možné vytvárať, čítať, modifikovať, či mazať dáta v databáze. \\

\noindent Databázový klient je generovaný do zložky \texttt{node\_modules}, odkiaľ je možné ho importovať ako z modulu \texttt{@prisma/client}. \\

\begin{lstlisting}[caption=Príklad vytvorenia instancie databázového klienta v Slackify.]
	import { PrismaClient } from "@prisma/client";

	export const prisma = new PrismaClient();
\end{lstlisting}



% Implementácia GraphQL služieb
%---------------------------------------------------------------------------
\chapter{Implementácia GraphQL služieb}
Ako bolo vysvetlené v sekcií \ref{design:architecture} o návrhu architektúry, redakčný systém Slackify disponuje dvomi GraphQL službami -- skrytou \texttt{service-private} a verejnou \texttt{service-public}. Obe GraphQL služby sú implementované pomocou balíčka Apollo Server.

% Schéma
\section{Schéma}
GraphQL schémy oboch služieb sú implementované pomocou balíčka \texttt{Nexus Schema}, ktorý umožňuje vytvoriť GraphQL schému v jazyku JavaScript namiesto SDL\footnote{SDL -- Schema Definition Language} definovanom v GraphQL špecifikácií. Vďaka tomu je možné budovať schému \uv{code-first} a silno typovanú. \\

\begin{lstlisting}[caption=Časť GraphQL schémy služby \texttt{service-private}., label={impl:code:schema}]
	export const Query = queryType({
		definition(t) {
			/* Users */
			t.crud.users({
				filtering: {
					team: true,
				},
			});
		}
	});
\end{lstlisting}

\noindent Balíček \texttt{Nexus Schema} disponuje aj pluginom, ktorý ho dokáže prepojiť s \texttt{prisma} klientom. Toto prepojenie rozširuje triedu \texttt{t} (prvý parameter v bloku \texttt{definition()}) o dve položky:

\begin{itemize}
	\item \texttt{t.model} -- Mapovanie položiek databázových modelov na GraphQL typy.
	\item \texttt{t.crud} -- Databázové operácie automaticky generované pre každý databázový model. Operácie podporujú stránkovanie, filtrovanie alebo zoraďovanie výsledkov. Vo výpise \ref{impl:code:schema} je znázornená operácia na získanie listu všetkých užívateľov s možnosťou filtrácie podľa tímu.
\end{itemize}

\noindent Plugin pre prepojenie schémy s \texttt{prisma} klientom však nepodporuje kontrolu autorizácie užívateľa, tá je riešená použitím \uv{middleware} funkcií.

% Autorizácia
\section{Autorizácia}
Každá požiadavka a mutácia (s výnimkou mutácie \texttt{signIn()} služby \texttt{service-private}) musí obsahovať HTTP hlavičku s názvom \texttt{Authorization}. Obe GraphQL služby však vyžadujú rozdieľne hodnoty tejto hlavičky, bližšie popísané v sekciách \ref{impl:service-private} a \ref{impl:service-public}.

\subsection{GraphQL Shield}
Skrytá aj verejná GraphQL služba pre kontrolu prístupu využíva aj middleware \texttt{GraphQL Shield}, ktorý umožňuje vytvoriť \uv{pravidlá} pre prístup ku konkrétnym požiadavkám alebo mutáciam. \\

\begin{lstlisting}[caption={Pravidlo \texttt{GraphQL Shield} kontrolujúce či je užívateľ autentifikovaný.}, label={impl:code:rule}]
	export const isAuthenticated = rule({ cache: 'contextual' })(
		async (_parent, _args, { user }: Context) => {
			return user !== undefined;
		}
	);
\end{lstlisting}

\medskip

\noindent Podobné pravidlá ako vo výpise \ref{impl:code:rule} sú využité pre kontrolu prístupu v celej GraphQL schéme. Ak niektorá z podmienok pravidla nie je splnená, dané pravidlo obvykle vráti instanciu typu \texttt{Error} so špecifickou správou, ktorá je zaslaná späť do webovej konzoly. Ak sú však splnené \emph{všetky podmienky}, pravidlo vracia boolean hodnotu \texttt{true} a riadenie je prenechané príslušnej GraphQL \texttt{resolver} funkcií.

% Skrytá služba service-private
\section{Skrytá služba \texttt{service-private}}
\label{impl:service-private}
Skrytá GraphQL služba (\texttt{service-private}) poskytuje prístup k dátam pre webové konzolu (službu \texttt{frontend}, ktorej implementácia je popísaná v sekcií \ref{impl:frontend}). Schéma tejto služby je tvorená prevažne CRUD operáciami nad databázovými modelmi \texttt{Collection} a \texttt{Component}.

\subsection{Autorizácia}
HTTP hlavička \texttt{Authorization} požiadavky alebo mutácie obsahuje JWT konkrétneho užívateľa. Tento token obsahuje okrem iných informácií o užívateľovi aj jeho ID. Vďaka tomu je možné pred vykonaním každej operácie získať všetky dostupné informácie o užívateľovi a jeho tíme, ktoré sú následne dostupné v kontexte každej GraphQL \texttt{resolver} funkcií. Ak nebolo možné nájsť užívateľa s daným ID, dáta o ňom a jeho tíme majú v GraphQL \texttt{resolver} funkcií hodnotu \texttt{undefined}.

% Verejná služba service-public
\section{Verejná služba \texttt{service-public}}
\label{impl:service-public}
Verejná GraphQL služba (\texttt{service-public}) slúži ako hlavné aplikačné rozhranie pre headless redakčný systém Slackify. Tento GraphQL server umožňuje vývojárom využívajúcim Slackify vo svojich implementáciach jednoducho získať obsah z redakčného systému.

\subsection{Autorizácia}
Po prepojení tímu v aplikácií Slack s redakčným systémom Slackify je tomuto tímu vygenerovaný \emph{unikátny autorizačný token}. Tento token je následne očakávaný v HTTP hlavičke \texttt{Authorization} každej požiadavky smerujúcej na službu \texttt{service-public}. V prípade, že autorizačný token chýba alebo je nesprávny, služba požiadavku zahodí a odpovie chybovou správou vysvetlujúcou zlyhanie. 

\subsection{Zjednodušený typ \texttt{Component}}
Databázový model \texttt{Component} slúži pre uchovávanie metadát o komponente. Samotný obsah komponentu sa nachádza v oddelených databázových modeloch s ktorými je daný komponent v relácií (viac viz \ref{impl:model:component}). Takýto princíp ukladania dát umožňuje jednoduchú modifikáciu typov komponentov a ich obsahu, no nie je optimálny pre vývojárov, ktorí implementujú Slackify do svojich projektov. \\

\begin{lstlisting}[caption={Príklad získania obsahu komponentu pred optimalizáciou.}]
	query {
		component(...) {
			id
			type // ??
			plainTextData
			articleData
			linkData
			...
		}
	}
\end{lstlisting}

\medskip

\noindent Vývojár musí totiž vopred poznať typ komponentu a podľa toho si vyžiadať z GraphQL servera konkrétnu reláciu s obsahom komponentu. Väčši problém však nastane v prípade, že vývojár vopred nepozná typ komponentu, ktorý žiada od GraphQL servera. V takejto situácií je nutné vyžiadať relácie s obsahmi pre \emph{všetky typy komponentov} a po obdržaní dát pristúpiť k správnemu obsahu podľa typu. \\

\begin{lstlisting}[caption={Definícia union typu pre dáta komponentu.}, label={impl:service-public:union}]
	export const ComponentData = unionType({
		name: 'ComponentData',
		definition(t) {
			t.members('PlainTextComponentData', 'ArticleComponentData', 'LinkComponentData');
			t.resolveType((data) => {
				// ...
			});
		},
	});
\end{lstlisting}

\medskip

\noindent Z tohto dôvodu dôvodu služba \texttt{service-public} disponuje upraveným typom \texttt{Component}, ktorý má špeciálne pole \texttt{data} typu \texttt{union} (viz výpis \ref{impl:service-public:union}). Toto pole združuje relácie s obsahom pre všetky typy komponentov. Ak si vývojár vyžiada toto pole, získa obsah komponentu bez ohľadu na to, akého je typu.

% Implementácia Slack aplikácie
%---------------------------------------------------------------------------
\chapter{Implementácia Slack aplikácie}
Služba \texttt{service-slack} je Slack aplikácia umožňujúca \emph{plnohodnotnú správu obsahu} v redakčnom systéme Slackify. Jedná sa o HTTP server implementovaný pomocu balíčka Bolt od spoločnosti Slack, ktorý reaguje na udalosti (events) a akcie (actions) prichádzajúce zo Slack API.

% Domovská stránka
\section{Domovská stránka}
Každá Slack aplikácia má k dispozícií jednu stránku, ktorej obsah môže prispôsobiť danému užívateľovi. Táto stránka má názov \uv{App Home} a v Slackify slúži ako jediné miesto, kde môže užívateľ spravovať obsah redakčného systému. Pri každej návšteve tejto stránky užívateľom Slack API odošle udalosť \texttt{app\_home\_opened} na službu \texttt{slackify-slack}. Po obrdžaní udalosti služba vygeneruje rohranie pre daného užvateľa a odošle ho späť na Slack API, ktoré sa postará aby bolo správne zobrazené užívateľovi v aplikácií. Rozhranie domovskej stránky je \emph{perzistentné}. \\

% TODO: Add app home screenshot

\noindent Na domovskej stránke sa na chádza hlavička so všetkými akciami, ktoré môže užívateľ v redakčnom systéme vykonávať (ak užívateľ nemá dostatočné práva, akcie sú skryté). Pod touto hlavičkou sa nachádza zoznam komponentov. Zoznam zobrazuje komponenty priradené k zvolenej kolekcií pomocou elementu \texttt{Select} v záhlaví zoznamu. Ak nie je zvolená žiadna kolekcia, zobrazuje komponenty priradené k prvej dostupnej kolekcií. \\

\begin{lstlisting}[caption={Funkcia zodpovedná za generovanie rozhrania domovskej stránky.}]
	async function compose_app_home_view(
		teamId: string,
		userId: string,
		initialCollectionId?: string
	): Promise<View | undefined>
\end{lstlisting}

\medskip

\noindent Domovská stránka je aktualizovaná po akciách užívateľa. Ak užívateľ vytvorí, odstráni alebo upraví kolekciu alebo komponent, rozhranie domovskej stránky musí byť aktualizované s novými zmenami.

% Dialógové okná
\section{Dialógové okná}
Dialógové okná v Slack aplikácií slúžia pre zobrazovanie informácií, ktoré sa nenachádzajú na domovskej stránke (napríklad zoznam kategórií) alebo pre získanie dát od užívateľa pomocou formulárov (napríklad vytvorenie novej kolekcie).

% Implementácia webovej konzoly
%---------------------------------------------------------------------------
\chapter{Implementácia webovej konzoly}
\label{impl:frontend}
Webová konzola (služba \texttt{frontend}) slúži pre komplexnejšiu správu obsahu v Slackify redakčnom systéme. Samotné uživateľské rozhranie je implementované pomocou JavaScriptovej knižnice React a frameworku Next.js, ktorý zabezpečuje vykresľovanie (render) na strane servera.

% Prihlásenie
\section{Prihlásenie}
Slackify nemá vlastný proces vytvorenia užívateľského účtu, naopak všetky \emph{dáta o užívateľoch preberá z aplikácie Slack} pomocou protokolu OAuth 2.0. \\

\noindent Proces príhlásenia užívateľa do Slackify cez aplikáciu Slack je následovný:

\begin{enumerate}
	\item Užívateľ, ktorý sa chce prihlásiť do Slackify je po kliknutí na tlačidlo \uv{Sign in with Slack} presmerovaný na autorizačnú stránku aplikácie Slack.
	\item Užívateľ vyplní svoje prihlasovacie údaje pre účet v tom pracovnom prostredí, \emph{ktoré má nainštalovanú aplikáciu Slackify}.
	\item Po úspešnom prihlásení užívateľ \emph{povolí Slackify prístup k užívateľským údajom} a je následne presmerovaný na stránku vo webovej konzole Slackify \texttt{/auth/redirect} s URL parametrom \textit{code}.
	\item Slackify odošle hodnotu URL parametra \textit{code} na OAuth API aplikácie Slack a očakáva odpoveď s \emph{autorizačným kódom pre daného užívateľa} a informáciach o ňom.
	\item Autorizačný token je následne možné využiť pre opätovné získanie informácií o užívateľovi alebo k vykonávaniu akcií za užívateľa.
\end{enumerate}

\noindent Po úspešnom obdržaní autorizačného tokena a užívateľských dát systém skontroluje, či už daný užívateľ v databáze existuje. V prípade, že užívateľ neexistuje, je záznam o ňom pridaný do databázy. V prípade, že už existuje, informácie o ňom sú len aktualizované. \\

\noindent Vygenerovaný JWT\footnote{JWT -- JSON Web Token} so základnými informáciami o užívateľovi je uložený do cookies prehliadača. Pri návšteve každej stránky prebieha kontrola, či uložený JWT nie je expirovaný. Napríklad pravidlo 

% Správa kolekcií
\section{Správa kolekcií}
Hlavnou súčasťou správy kolekcií je zoznam všetkých kolekcií, ktoré boli vytvorené v tíme aktuálne prihláseného užívateľa. Na tejto stránke je možné kolekcie vytvoriť, upraviť alebo zmazať, ak má užívateľ na tieto akcie dostatočné práva.

\begin{figure}[h]
	\centering
	\includegraphics[scale=0.085]{obrazky-figures/screenshot_collections}
	\caption{Stránka správy kolekcií vo webovej konzole.}
\end{figure}

\noindent Pri úvodnom načítaní stránky je odoslaná GraphQL požiadavka (query) \texttt{collections()}, ktorá získa prvých 40 kolekcií zo služby \texttt{service-private}. Ak chce užívateľ zobraziť viac kolekcií, musí zísť na koniec zoznamu. Na jeho konci sa nachádza element, ktorý ak je viditeľný v okne prehliadača odošle znova požiadavku \texttt{collections()}, ale tentoraz s priloženými parametrami \texttt{skip} a \texttt{first} (význam parametrov bližšie popísaný v sekcií \ref{?}). Získané položky sú pridané na koniec zoznamu a takto je vytvorený mechanizmus tzv. \uv{nekonečného scrollovania} (infinite scrolling).
% TODO: Add a ref for query collections() section

\subsection{Vytvorenie a úprava kolekcie}
Vytvorenie novej kolekcie alebo úprava už existujúcej je možná pomocou dialógového okna, ktoré je dostupné na stránkach zoznamu a detailu kolekcií. Objekt uložený v globálnom stave aplikácie pod názvom \texttt{createUpdateModal} uchováva kontext dialógového okna, ktorý sa obsahuje položky:

\begin{itemize}
	\item \texttt{mode} -- Mód zobrazenia, môže nadobudnúť hodnoty \uv{create} alebo \uv{update}.
	\item \texttt{collection} -- V prípade, že mód zobrazenia má hodnotu \uv{update}, očakáva sa, že hodnota tejto položky obsahuje objekt popisujúci upravovanú kolekciu.
\end{itemize}

\noindent Po potvrdení formulára vytvorenia alebo úpravy kolekcie je na službu \texttt{service-private} odoslaná mutácia \texttt{createOneCollection()}, resp. \texttt{updateOneCollection()} s dátami z formulára. Ak bola mutácia úspešná, v odpovedi od služby \texttt{service-private} by sa mali nachádzať dáta novej, resp. aktualizovanej kolekcie. Tieto dáta sú následne použité pre aktualizovanie cache webovej aplikácie.

\begin{figure}[h]
	\centering
	\includegraphics[scale=0.085]{obrazky-figures/screenshot_collection_create}
	\caption{Dialógové okno pre vytvorenie a úpravu kolekcií.}
\end{figure}

% Správa komponentov
\section{Správa komponentov}
Podobne ako u kolekcií je hlavnou súčasťou správy komponentov zoznam všetkých komponentov, ktoré boli vytvorené v tíme aktuálne prihláseného užívateľa. Podmienkou vytvorenia nového komponentu je už \emph{aspoň jedna existujúca kolekcia}, ku ktorej môže byť daný komponent priradený. Na tejto stránke je možné okrem vytvorenia nového komponentu aj upraviť alebo zmazať už existujúce komponenty.

\begin{figure}[h]
	\centering
	\includegraphics[scale=0.085]{obrazky-figures/screenshot_components}
	\caption{Stránka správy komponentov vo webovej konzole.}
\end{figure}

\noindent Rovnako ako pri zozname kolekcií je pri úvodnom načítaní odoslaná požiadavka pre získanie z ktorých sa zoznam zostaví. Pre komponenty je táto požiadavka pomenovaná \texttt{components()}. Zoznam komponentov takisto podporuje aj infinite scrolling mechanizmus.

\subsection{Vytvorenie a úprava komponentov}
Rovnaký princíp ako pri tvorbe a úprave kolekcií. Jedná sa o dialógové okno, ktoré svoj kontext uchováva v globálnom stave aplikácie, avšak mierne pozmenený:

\begin{itemize}
	\item \texttt{mode} -- Mód zobrazenia, rovnako ako u kolekcií môže nadobudnúť hodnoty \uv{create} alebo \uv{update}.
	\item \texttt{collection} -- Obsahuje informácie o aktuálne zvolenej kolekcií pri tvorbe nového komponentu. Typ kolekcie určuje typ k \emph{nej priradených komponentov}, a preto je táto kolekcia využitá pre zobrazenie relevantného formulára pre konkrétny typ komponentov.
	\item \texttt{component} -- V prípade, že mód zobrazenia má hodnotu \uv{update}, očakáva sa, že hodnota tejto položky obsahuje objekt poposijúci upravovaný komponent.
\end{itemize}

\begin{figure}[h]
	\centering
	\includegraphics[scale=0.085]{obrazky-figures/screenshot_component_update}
	\caption{Dialógové okno pre vytvorenie a úpravu komponentov.}
\end{figure}

\noindent Po povrdení formulára je pre vytvorenie nového komponentu alebo úpravu už existujúceho odoslaná mutácia \texttt{createOneComponent()}, resp. \texttt{updateOneComponent()} na službu \texttt{service-private}.

% Zmena role užívateľa
\section{Zmena role užívateľa}
Každý užívateľ môže navštíviť stránku so zoznamom všetkých užívateľov tímu, ktorého je členom. V prípade, že má aktuálne prihlásený užívateľ rolu \uv{Owner}, môže navyše meniť role ostatných užívateľov v rámci tímu. U každého užívateľa v zozname sa nachádza HTML element \texttt{<Select />}, ktorý ukazuje aktuálnu rolu daného užívateľa. Pri zmene na inú hodnotu tento element odošle mutáciu \texttt{updateOneUser()} so zvolenou hodnotou na službu \texttt{service-private}, ktorá aktualizuje rolu daného užívateľa.

\chapter{Testovanie}
